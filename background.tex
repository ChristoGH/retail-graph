
\subsection{Background}
The spend data has been extracted and made available in csv format.The data includes the following headers.  Each row is represents a single transaction.
\subsection{Naming explanation}
\begin{itemize}
  \item \textbf{\_id} A unique identifier from the MongoDB collections.
  \item \textbf{Dedupegroup} A client number, for lookups in the datawarehouse.
  \item \textbf{Seg\_L3\_Num} Level 3 client segmentation number.
   \item \textbf{Seg\_L3\_STR} Level 3 client segmentation dscription.
  \item \textbf{TransactionAmount} The amount of the transaction, negative for money flowing from an account.  Inflows are such as refunds are positive.
  \item \textbf{TransactionDate} Date of transaction format \newline 'yyyy-mm-ddTHH:MM:SS.fffZ'
  \item \textbf{dayname} String description of day, ie Sunday, Monday etc.
  \item \textbf{period} The period of transaction, yyyyMM.
  \item \textbf{month} A client number, for look ups in the data-warehouse.
  \item \textbf{universaldate} A retail short date.  The first Sunday of every month is labelled the 1st, the first Monday of every month is labelled the 2nd etc.
  \item \textbf{weekday} Integer representing the day of the week.
  \item \textbf{weekth} The week of the year number.
  
  \item \textbf{companyindex} A unique identifier for classified companies.
  \item \textbf{companyname} A unique string name for classified companies.
  \item \textbf{franchisename} A unique string name for a merchant.
  
  \item \textbf{class\_id} Class identifier of company.
  \item \textbf{subclass\_id} Sub-class identifier of company.
  \item \textbf{group\_id} Group identifier of company.
  \item \textbf{division\_id} Division identifier of company.
  \item \textbf{discretionary} Classifying the type of spend at a classified company.
  \item \textbf{channel} Transaction mode description.

  \item \textbf{db} Name of source db.
  \item \textbf{dbprefix} Name of source table.
\end{itemize}


\subsubsection{Name convention}
Enforce a standard parameter naming convention.  Where possible we keep the origin names.  Small caps an underscore word separation:

\begin{center}
 \begin{tabular}{||r l||} 
 \hline
 Import & Neo4J  \\ [0.5ex] 
 \hline\hline
 Dedupegroup & dedupestatic \\ 
 Seg\_L3\_Num & seg\_l3\_num \\
 Seg\_L3\_STR & seg\_l3\_str \\
 TransactionAmount & transactionamount \\
 TransactionDate & transactiondate \\
 dayname & dayname \\ 
 period & period \\ 
 month & month \\ 
 
 universaldate & universaldate \\ 
 weekday & weekday \\
 weekth & weekth \\
 companyindex & companyindex \\
 companyname & companyname \\ 
 franchisename & franchisename \\  
 class\_id & class\_id \\ 
 subclass\_id & subclass\_id \\ 
 group\_id & group\_id \\
 division\_id & division\_id \\
 discretionary & discretionary \\ 
 channel & channel \\ 
 db & db \\ 
 dbprefix & dbprefix \\
 \hline
\end{tabular}
\end{center}


The retail graph consists of two nodes, client and merchant. 

In the graph database we enforce uniqueness constraints on the franchise name of the merchant node and the client identifier on the client node. 

All transactions for the period March 2020 was loaded into a Neo4j graph database. For the purpose of this work we only look at one subset of merchant with subclass\_id = 56101.  These are known as the 'fast food' brands or formally 'Food service activities of take away counters', loosely as some offer sitdown services, blurring the line between a restaurant and a entity known traditionally as a fast food restaurant.